\section{Introduction}
\label{sec:introduction}

Current mobile devices allow us to listen to any music at any time and any place.
(Music) Online stores offer billions of song tracks, but which ones to opt for?

Recommending song tracks to users in some way is indeed an important aspect for a variety of reasons:
Music online stores try to recommend tracks to users which they both did not purchase before but might be interested in, with the aim to increase their revenue.
Collaborative listening portals like last.fm or Pandora automatically generate music playlists for some given seed song, and provide the user with a sequential and unsteerable list of tracks they can listen to.
One of their aims is to get users in contact to novel tracks they did not know before, again with the aim to sell those.
%Their aim is closely related to our work here: Providing the user with related music content, based on personal listening preferences.
Portable devices are often used to listen to some own music while doing something else, e.g., sports.
`Automatic playlist generation', a sense-making way of generating a possible sequence or set of songs to play (next), given some fixed, but potentially huge set of songs in a library, is involved in all of the cases above.

While all these tasks are of the same kind, the available data for the tasks differs considerably:
Online stores and listening portals have usually no complete information about the music library of the user, and, if portable devices were able to automatically generate playlists, they are only able to choose next tracks out of the users' music library.

In this paper, we exploit massive collaborative usage data gathered on portable devices, to recommend the user songs to listen to next.
We both recommend songs to the user which are either already existing or not existing in his library.
As an applicational scenario, we think of a mobile player which is able to combine the user library with some foreign music store, to allow the user to either (automatically) purchase recommended songs, or continue listening to her/his own songs.

The rest of the paper is organized as follows.
Section 2 reviews some related literature.
We formally define the problem and possible approaches in section 3.
Our evaluation in section 4 is focused on both the 'next-song' and 'following-songs' recommendation task.
Section 5 concludes.



%We here consider the problem of 'next-song-recommendation' in a scenario, where only playlogs and song metadata exist.
%Having an anonymous users setting in mind, our task is to recommend next songs based on both the users previous listening history data, and collaborative listening data extracted from logfiles from Microsofts' Zune player.
%
%playlists available
%
%lots of collaborative information -> last.fm

