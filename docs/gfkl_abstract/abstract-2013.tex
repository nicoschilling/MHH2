%**********************************************************
% Guidelines for abstracts for the European Conference on Data Analysis 2013
% of the GfKl anf the SFC
%**********************************************************
%
\documentclass{svmult}

\usepackage{amsmath}
\usepackage{amsfonts}
\usepackage{latexsym}
\usepackage{graphicx}
\usepackage{multicol}

\begin{document}

\title*{Guidelines for Authors of the\protect\linebreak
Abstract Volume of the\protect\linebreak European Conference on Data Analysis 2013}

\author{Name of Author\inst{1}\and
Name of Author\inst{2}}
% Use \authorrunning{Short Title} for an abbreviated version of
% your contribution title if the original one is too long
\institute{Name and Address of your Institute
\texttt{name@email.address} \and Name and Address of your Institute
\texttt{name@email.address}}
%
% Use the package "url.sty" to avoid
% problems with special characters
% used in your e-mail or web address
%
\maketitle

\begin{abstract}
To obtain a standardized layout when printing the abstract volume,
we ask all authors to use the \LaTeX\ word processor and to take
notice of the following instructions. The text must be written in
English and must not exceed one page. 

Please use this file for writing your abstract and do not use 
different or additional macro-packages.
This file takes care of all the
formatting. Please, format title, author(s), affiliation(s), text of
the abstract, references, and 3--5 keywords according to this sample
\LaTeX\ file. The \LaTeX\ file of the abstract should be submitted
online via the conference web sites:
%
\begin{center}
{\tt http://www.gfkl2013.lu/} or {\tt http://www.sfc2013.lu/}
\end{center}
%
The\begin{bf} deadline\end{bf} for submission of the abstracts is\begin{bf}
February 28th, 2013\end{bf}. Accepted abstracts will be included in
a summary volume that will be distributed to conference
participants.
\end{abstract}

\begin{thebibliography}{1}

\item[]
% Example for the citation of articles:
BAIER, D. and GAUL, W. (1999): Optimal Product Positioning Based on
Paired Comparison Data. {\em Journal of Econometrics, 89, 365--392}.

\item[]
% Example for the citation of books:
FLACH, P.A. (2012): {\em Machine Learning: The Art and Science of Algorithms that Make Sense of Data}. Cambridge University Press, New York.

\item[]
% Example for the citation of papers in proceedings volumes:
BRUSCH, M. and BAIER, D. (2002): Conjoint Analysis and Stimulus
Presentation: a Comparison of Alternative Methods. In: Zoubin Ghahramani (Ed.): {\em Proceedings of the 24th international conference on Machine learning}. ACM, New York, NY, USA, 681--688.

\end{thebibliography}

\section*{Keywords}
ABSTRACTS, GUIDELINES, LAYOUT, REFERENCES

\end{document}
