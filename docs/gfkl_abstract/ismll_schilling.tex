%**********************************************************
% Guidelines for abstracts for the European Conference on Data Analysis 2013
% of the GfKl anf the SFC
%**********************************************************
%
\documentclass{svmult}

\usepackage{amsmath}
\usepackage{amsfonts}
\usepackage{latexsym}
\usepackage{graphicx}
\usepackage{multicol}

\begin{document}

\title*{Event Prediction in Pharyngeal High-Resolution Manometry }

\author{Nicolas Schilling\inst{1}\and Andre Busche\inst{1}\and
Simone Miller\inst{2}\and Michael Jungheim\inst{2}\and Martin Ptok\inst{2}\and Lars Schmidt-Thieme\inst{1}}
% Use \authorrunning{Short Title} for an abbreviated version of
% your contribution title if the original one is too long
\institute{University of Hildesheim, Information Systems and Machine Learning Lab
\texttt{\{schilling,busche,schmidt-thieme\}@ismll.uni-hildesheim.de} \and Medizinische Hochschule Hannover, Klinik f\"{u}r Phoniatrie und P\"{a}daudiologie
\texttt{\{Miller.Simone,Jungheim.Michael,Ptok.Martin\}@mh-hannover.de}}
%
% Use the package "url.sty" to avoid
% problems with special characters
% used in your e-mail or web address
%
\maketitle

\begin{abstract}
A prolonged phase of increased pressure in the upper esophageal sphincter (UES) after swallowing might result in globus sensation. Therefore it is mandatory to evaluate restitution times of the UES in order to distinguish normal from impaired swallow associated muscular activities. Estimating the event $t^{\star}$ where the UES has returned to its resting pressure after a patient has swallowed can be accomplished by predicting if swallowing activities are present or not. While the problem, whether a patient suffers from dysphagia, is approached in [MIELENS, 2012], our task directly allows for a profound understanding of pharyngeal activities.

From the machine learning point of view, the problem can be understood as binary sequence labeling, while the goal is not to globally find the most accurate sequence classification, but to find a sample $t^{\star}$ within the sequence obeying a certain characteristic: We seek a best approximation of label change which can be thought of as a dissection of the sequence into two individual parts. Whereas commonly used models for sequence labeling are based on graphical models [NGUYEN, 2007], we approach the problem using logistic regression as classifier and integrating sequential features by means of FFT-coefficients. Additionally, we add a Laplacian regularizer in order to encourage a smooth classification due to the monotonicity of target labels.

% We tackle the problem of binary sequence labeling, while in our applicational scenario, the goal is not to globally find the most accurate sequence classification, but to find a single sample $t^{\star}$ within the sequence obeying a certain characteristic: From the monotonicity of target labels, we seek a best approximate of label change, which can be thought of as a dissection of the sequence into two individual parts. Whereas commonly used models for sequence labeling are based on graphical models, we approach the problem using logistic regression as classifier and integrating sequential features by means of FFT-coefficients. 
%From the intrinsic monotonicity of target labels,
% As medical application, the aim is to estimate the point of time $t^{\star}$ where all the pharyngeal activities have subsided after a Patient has swallowed. In this sense, the aim is to predict, whether a Patient is swallowing or not, which then can be used to deduce $t^{\star}$.
%Moreover, we augment the classification by a Laplacian smoothing, due to the fact that the classification throughout a sequence should change only twice.
% From domain knowledge, we add a Laplacian regularizer in order to encourage a smooth classification, due to the fact that the labels throughout a sequence should change only once.
% While the problem, whether a Patient suffers from dysphagia, is approached in [MIELENS, 2012], our task is to identify the duration of a Patient's swallow, which directly allows for a profound understanding of pharyngeal activities.

\end{abstract}

\begin{thebibliography}{1}

\item[]
% Example for the citation of articles:
MIELENS, J. et al. (2012): Application of Classification Models to Pharyngeal High-Resultion Manometry. {\em Journal of Speech, Language, and Hearing Research, 55, 892--902}.

\item[]
% Example for the citation of papers in proceedings volumes:
NGUYEN, N. and GUO, Y. (2007): Comparison of Sequence Labeling Algorithms and Extensions. In: Zoubin Ghahramani (Ed.): {\em Proceedings of the 24th international conference on Machine learning}. ACM, New York, NY, USA, 681--688.

\end{thebibliography}

\section*{Keywords}
Sequence Labeling, Laplacian Regularizer, Pharyngeal Manometry

\end{document}


%This problem can also be approached through a time series regression, while our preliminary research focuses on the former approach
